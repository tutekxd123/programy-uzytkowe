\documentclass[a4paper]{article}
\author{Piotr Marzec 3 GRP}
\title{Ćwiczenia nr.3 Latex}
\usepackage[left=3.5cm, right=2.5cm, top=2.5cm, bottom=2.5cm]{geometry}
\usepackage[MeX]{polski}
\usepackage[utf8]{inputenc}
\usepackage{graphicx}
\usepackage{enumerate}
\usepackage{amsmath} %pakiet matematyczny
\usepackage{amssymb} %pakiet dodatkowych symboli
\usepackage{float}
\begin{document}
\begin{center}
    \begin{table}[H]
        \centering\caption{Przykładowy system decyzyjny(U,A,d),modelujący problem diagnozy medycznej, której efektem jest decyzja o wykonaniu lub nie wykonaniu operacji wycięcia wyrostka robaczkowego, u=\{$u_1,u_2....u_10$\},a=\{$a_1,a_2$\},$d\in D$=\{TAK,NIE\} }
        \begin{tabular}{c|c|c|c}
             \hline
             \hline
           Pacjent & Ból brzucha & Temperatura ciała & Operacja\\
           \hline
           u1 & Mocny & Wysoka & Tak\\
           u2 & Średn & Wysoka & Tak\\
           u3 & Mocny & Średnia & Tak\\
           u4 & Mocny & Niska & Tak \\
           u5 & Średni & Średnia & Tak \\
           u6 & Średni & Średnia & Nie \\
           u7 & Mały & Wysoka & Nie \\
           u8 & Mały & Niska & Nie \\
           u9 & Mocny & Niska & Nie \\
           u10 & Mały & Średnia & Nie \\
           \hline
           \hline
        \end{tabular}
    \end{table}
    \begin{table}[H]
        \centering\caption{Bramka NOT}
        \begin{tabular}{|c|c|}
            \hline
             x&y  \\
             \hline
             0&1 \\
             \hline
             1&0 \\
             \hline
        \end{tabular}
    \end{table}
       \begin{table}[H]
        \centering\caption{Bramka AND}
        \begin{tabular}{|c|c|c|}
            \hline
             x&y&result  \\
             \hline
             0&1&0 \\
             \hline
             1&0&0 \\
             \hline
             0&0&0\\
             \hline
             1&1&1\\
             \hline
             \end{tabular}
    \end{table}
\begin{table}[H]
        \centering\caption{Bramka NAND}
        \begin{tabular}{|c|c|c|}
            \hline
             x&y&result  \\
             \hline
             0&1&1 \\
             \hline
             1&0&1 \\
             \hline
             0&0&1\\
             \hline
             1&1&0\\
             \hline
             \end{tabular}
    \end{table}
\begin{table}[H]
        \centering\caption{Bramka OR}
        \begin{tabular}{|c|c|c|}
            \hline
             x&y&result  \\
             \hline
             0&1&1 \\
             \hline
             1&0&1 \\
             \hline
             0&0&0\\
             \hline
             1&1&1\\
             \hline
             \end{tabular}
    \end{table}
\begin{table}[H]
        \centering\caption{Bramka NOR}
        \begin{tabular}{|c|c|c|}
            \hline
             x&y&result  \\
             \hline
             0&1&0 \\
             \hline
             1&0&0 \\
             \hline
             0&0&1\\
             \hline
             1&1&0\\
             \hline
             \end{tabular}
    \end{table}
\begin{table}[H]
        \centering\caption{Bramka XOR}
        \begin{tabular}{|c|c|c|}
            \hline
             x&y&result  \\
             \hline
             0&1&1 \\
             \hline
             1&0&1 \\
             \hline
             0&0&0\\
             \hline
             1&1&0\\
             \hline
             \end{tabular}
    \end{table}
\end{center}

\end{document}
