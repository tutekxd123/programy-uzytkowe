\documentclass[a4paper]{article}
\author{Piotr Marzec 3 GRP}
\title{Ćwiczenia nr.1 Latex}
\usepackage[left=3.5cm, right=2.5cm, top=2.5cm, bottom=2.5cm]{geometry}
\usepackage[MeX]{polski}
\usepackage[utf8]{inputenc}
\usepackage{graphicx}
\usepackage{enumerate}
\usepackage{amsmath} %pakiet matematyczny
\usepackage{amssymb} %pakiet dodatkowych symboli
\begin{document}
\maketitle
\newpage
\section{Historia systemów}
\begin{enumerate}
	\item Windows
		\begin{enumerate}[I.)]
			\item XP
			\item 7
			\item 8.1
			\item 10
		\end{enumerate}
	\item GNU/Linux
		\begin{enumerate}
			\item Ubuntu
			\begin{enumerate}[I.)]
				\item 18.04LTS
				\item 19.10
				\item 20.04LTS
			\end{enumerate}
			\item Debian
				\begin{enumerate}[I.)]
					\item 8
					\item 9
					\item 10
				\end{enumerate}
		\end{enumerate}
\end{enumerate}
\newpage
\section{Przepis na placki ziemniaczane}
\subsection{Składniki:}
\begin{enumerate}
	\item 1/2 kg ziemniaków
	\item 1/2 łyżki mąki pszennej
	\item 1/4 cebuli
	\item 1 jajko
	\item sól
	\item olej roślinny do smażenia
\end{enumerate}
\subsection{Przygotowanie}
\begin{enumerate}
	\item Ziemniaki obrać i zetrzeć na tarce o małych oczkach bezpośrednio do większej i płaskiej miski. Zostawić je w misce bez mieszania, miskę delikatnie przechylić i odstawić tak na ok. 5 minut.
	\item W międzyczasie odlewać zbierający się sok, delikatnie przytrzymując ziemniaki, nadal ich nie mieszać. Na koniec docisnąć dłonią do miski i odlać jeszcze więcej soku. Dodać mąkę, drobno startą cebulę, jajko oraz dwie szczypty soli.
	\item Rozgrzać patelnię, wlać olej. Masę ziemniaczaną wymieszać. Nakładać porcje masy (1 pełna łyżka) na rozgrzany olej i rozprowadzać ją na dość cienki placek. Smażyć na średnim ogniu przez ok. 2 - 3 minuty na złoty kolor, przewrócić na drugą stronę i powtórzyć smażenie.
	\item Odkładać na talerz wyłożony ręcznikami papierowymi. Posypać solą morską z młynka. Placki ziemniaczane najlepsze są prosto z patelni gdy są chrupiące.
\end{enumerate}
\end{document}